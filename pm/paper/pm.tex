\documentclass[twocolumn,10pt]{asme2ej}

\usepackage{epsfig} 

\title{%
	Personenzentrisches Projektmanagement\\
	\large 
	- \\
	Einfluss verschiedener Persönlichkeiten auf \\ 
	das Projektmanagement von Data Science Projekten}

\author{Tobias Rohrer
    \affiliation{
	Hochschule Darmstadt\\
	Data Science (Master)\\
    Email: sttorohr@stud.h-da.de
    }	
}

\graphicspath{ {./images/} }
\usepackage[center]{caption}
\usepackage[locale=DE]{siunitx}
\usepackage{hyperref}

\usepackage{listings}
\usepackage{xcolor}
\usepackage{enumitem}
\definecolor{codegreen}{rgb}{0,0.6,0}
\definecolor{codegray}{rgb}{0.5,0.5,0.5}
\definecolor{codepurple}{rgb}{0.58,0,0.82}
\definecolor{backcolour}{rgb}{0.95,0.95,0.92}

\lstdefinestyle{mystyle}{
	backgroundcolor=\color{backcolour},   
	commentstyle=\color{codegreen},
	keywordstyle=\color{magenta},
	numberstyle=\tiny\color{codegray},
	stringstyle=\color{codepurple},
	basicstyle=\ttfamily\footnotesize,
	breakatwhitespace=false,         
	breaklines=true,                 
	captionpos=b,                    
	keepspaces=true,                 
	numbers=left,                    
	numbersep=5pt,                  
	showspaces=false,                
	showstringspaces=false,
	showtabs=false,                  
	tabsize=2
}
\lstset{style=mystyle}
\begin{document}

\maketitle    

%%%%%%%%%%%%%%%%%%%%%%%%%%%%%%%%%%%%%%%%%%%%%%%%%%%%%%%%%%%%%%%%%%%%%%
\begin{abstract}


\end{abstract}

%%%%%%%%%%%%%%%%%%%%%%%%%%%%%%%%%%%%%%%%%%%%%%%%%%%%%%%%%%%%%%%%%%%%%%
\section{Einleitung}
Methoden und Werkzeuge zum Managen von Projekten werden meist anhand des Projektumfangs, Komplexität oder der Klarheit von Anforderungen ausgewählt. Außer acht gelassen wird dabei jedoch, dass sich ein Projektteam aus verschiedenen Persönlichkeiten mit unterschiedlichen Stärken und Schwächen zusammen setzt. Einer der Punkte im Agilen Manifesto lautet sogar "Individuals and interactions over processes and tools" \cite{beck2001agile}. Im Folgenden wird Diskutiert, wie sich verschiedene Projektmanagement-Werkzeuge und Methoden auf die Stärken und Schwächen, sowie auf die Motivation der Persönlichkeiten nach dem DISC-Modell auswirken.

Dazu wird in Abschnitt \ref{sec:1} zunächst das Projektmanagement in Data Science Projekten via Scrum, Kanban und dem Wasserfallmodell, sowie die darin verwendeten Projektmanagement-Werkzeuge gegenüber gestellt.

In Abschnitt \ref{sec:2} werden anschließend die einzelnen Persönlichkeitskategorien nach dem DISC-Modell beschrieben. Außerdem wird diskutiert, wie die Projektmanagement-Methoden und Werkzeuge aus Abschnitt \ref{sec:1} unter Berücksichtigung der Stärken und Schwächen der am Team teilnehmenden Persönlichkeiten eingesetzt werden sollten.


\section{Projektmanagement in Data Science Projekten}\label{sec:1}
Es gibt unzählige Methoden, um Projekte zu verwalten. Im Folgenden werden 3 der meist verbreitetsten kurz beschrieben\footnote{Ausführlichere Beschreibungen der Methoden in den Quellen.}. Anschließend werden die Besonderheiten von Data Science Projekten erläutert und die drei zuvor beschriebenen Methoden im Bezug auf diese Besonderheiten gegenüber gestellt.

\subsection{Wasserfallmodell}
Das Wasserfallmodell lässt sich am besten durch seine sequenziell ablaufenden Phasen charakterisieren. Eine neue Phase wird erst nach erfolgreichem Abschluss der vorgeschalteten Phase gestartet. Zwar bietet das Wasserfallmodell eine gute Übersicht über die Gesamtplanung des Projekts, doch ist es relativ unflexibel gegenüber Veränderungen. \cite{Wasserfall}

\subsection{Agile}
Scrum und Kanan gehören beide zur Gruppe der Agilen Methoden und teilen demnach deren Grundprinzipien. Diese sind im Agile Manifesto gestgehalgen und sind:
- Shortest Path. Reduce Wait times...
\cite{beck2001agile}


\subparagraph{Scrum}
Die Grundiee von Scrum ist die Einteilung des Projektes in Inkremente. Jedes Inkrement wird innerhalb eines sogenannten Sprints bearbeitet und anschließend ausgeliefert. Somit kann kontinuierlich Rückmeldung des Kunden in den Entwicklungsprozess einfließen. In einem Team, das nach Scrum arbeitet gibt es feste Rollen wie den Product Owner, den Scrum Master sowie das Projekt Team. Dem Product Owner obliegt üblicherweise die inhaltliche und dem Scrum Master die organisatorische Verantwortung.

\subparagraph{Kanban}
Angefangene Aufgaben schnellst möglich fertig zu stellen ist eines der zentralen Punkte von Kanban. Um dies zu unterstützten wird üblicherweise ein work in progress (WiP) limit definiert, um sich auf angefangene Aufgaben zu konzentrieren und deren Fertigstellung zu beschleunigen. Dieser Prozess wird auf einem gemeinsamen Kanban Board visualisiert. Hierauf werden die aktuellen Aufgaben, sowie deren Status dargestellt. Ein weiteres Ziel die kontinuierliche Effizienzsteigerung dieses Prozesses.\cite{kanban} Aktualisierungen werden an den Kunden ausgeliefert, sobald Sie fertig sind. In Kanban gibt es keine festen Teamrollen. Üblicherweise wird der Kanban Prozess durch die sogenannte Lead und Cycle Zeit überwacht. Diese misst die Zeit, die eine Aufgabe von der Aufnahme bis zur fertigstellung benötigt. Der Prozess, mit dem das Projektmanagement nach Kanban aufgebaut wird kann sich jederzeit ändern. Im Vergleich zu Scrum kann man Kanban als eher kontinuierliche Methode sehen, wobei Scrum auf feste Inkremente mit definiertem Start und Ende (Sprints) arbeitet.

\subsection{Besonderheiten von Data Science Projekten}
Bei Data Science Projekten ist zu Projektstart oft nicht genau abzuschätzen, ob und in wie fern die festgelegten Ziele mit den verfügbaren Daten erreicht werden können. Dies liegt unter anderem daran, dass die Qualität der zu verarbeitenden Daten fehl eingeschätzt wird. Diese wird in der Praxis eher über statt unterschätzt. Unter Anderem findet deshalb die Auswahl der einzusetzenden Technologien oft erst im laufe des Projektes statt \cite{agile_pm}. Darüber Hinaus können sich die Anforderungen im Laufe des Projektes ändern. Beispielsweise werden durch die Verarbeitung der Daten neue Möglichkeiten vom Entwicklungsteam aber auch von den "Kunden" überhaupt erst erkannt. 

\subsection{Die Auswahl einer passenden Methode}

\begin{figure}
	\includegraphics[scale=0.65	]{stacey.png}
	\caption[center]{Die Stacey Matrix angelehnt an \cite{stacey_img}}
	\label{fig:stacey}
\end{figure}

Eine Herausforderung zu beginn vieler Projekte stellt die Wahl einer passenden Projektmanagement Strategie da. Hierbei kann die Stacey Matrix \cite{Stacey2011StrategicMA} unterstützen, indem Projekte in Einfach, Kompliziert, Komplex oder Chaotisch klassifiziert werden (siehe Abbildung \ref{fig:stacey}). Die Klassifizierung wird anhand der Unklarheit und Komplexität der eingesetzten Technologien (Wie wird das Projekt umgesetzt), sowie der Klarheit über die Anforderungen an das Projekt (Was wird in dem Projekt umgesetzt) vorgenommen. Bei einer Klassifizierung im "oberen" komplizierten Bereich bis hin zum Chaotischen Bereich sollten Methoden aus dem Agilen Umfeld, in unserem Fall Kanban oder Scrum eingesetzt werden. Bei einer Klassifizierung als Einfach oder im unteren Komplizierten Bereich kann, muss aber nicht, auch das Wasserfallmodell verwendet werden. Durch die oben beschriebenen Unklarheiten bezüglich der Technologien sowie der Ziele sind die meisten Data Science Projekte in der Stacey Matrix eher ab der komplizierten Stufe einzuordnen. Das Wasserfallmodell ist für die meisten Data Science Projekte wegen der Inflexibilität gegenüber unvorhersehbaren kurzfristigen Abweichungen und Änderungen  eher ungeeignet.

-> Unternehmensstruktur !!

\section{DISG Projektmanagement}\label{sec:2}
Ein Projektteam ist dann effizient, wenn die Teilnehmenden motiviert sind und ihre Stärken voll einsetzen können. Es sollte dabei nicht überraschen, dass Personen unterschiedliche Stärken und Schwächen besitzen und sich durch verschiedene Dinge motivieren lassen. Im Folgenden werden die Persönlichkeitstypen nach dem DISG-Modell beschrieben. Des Weiteren wird dargelegt, wie sich die Projektmanagement Strategien aus Abschnitt \ref{sec:1} auf die Stärken, Schwächen und die Motivation der DISG-Persönlichkeitstypen auswirkt. 

\subsection{DISG-Modell}
Anhand des DISG-Modells \cite{disc} können Persönlichkeiten in die Bereiche Dominant(D), Initiativ(I), Stetig(S) und Gewissenhaft(G). Eine Persönlichkeit setzt sich meist aus unterschiedlichen Bereichen verschieden stark ausgeprägt zusammen. Mit Hilfe eines DISG-Persönlichkeitstests kann ein Persönlichkeitsprofil erstellt werden. Die verschiedenen Persönlichkeitstypen nach dem DISG-Modell haben typische Stärken und Schwächen, sowie Motivationsfaktoren. 

\subsection{Der Dominante}
Der Dominante  wird vor Allem durch Erfolge und Herausforderungen motiviert. Außerdem genießt der Dominante bei der Bewältigung seiner Aufgaben gerne den Freiraum um Risiken einzugehen und um Entscheidungen zu treffen oder zumindest auf diese einzuwirken. Seine größte Stärke ist die zielorientierte Arbeitsweise. Eine seiner Schwächen, die sich jedoch auch positiv auswirken kann, ist die Ungeduld. Außerdem kann seine direkte Art Konflikte im Team auslösen. Es besteht auch die Gefahr, dass der Dominante seine Autorität überschreitet. Zuletzt laufen dominante Persönlichkeiten die Gefahr sich bei der Aufgabenplanung zu überschätzen.

\subsubsection{Im Agilen Umfeld}
Eine neue Software Version an Kunden auszuliefern kann als Teilerfolg eines Teams angesehen werden. Das Prinzip der frühen und kontinuierlichen Auslieferung von Software könnte sich daher motivierend auf den Dominanten auswirken. Seine ungeduldige Eigenschaft kann sich sogar positiv darauf auswirken, diesem Prinzip treu zu bleiben. Ein weiteres Agiles Prinzip, Projekte um motivierte Personen aufzubauen, kann dem Dominanten den nötigen Rahmen geben, um mit seiner zielstrebigen Arbeitsweise das gesamte Team zu bereichern. Durch das Prinzip regelmäßige Reflektionsveranstaltungen durchzuführen, können Konflikten, die durch die direkte Art des Dominanten entstehen, direkt entgegen gewirkt werden.

\subparagraph{Kanban} Das Ziel, angefangene Aufgaben möglichst schnell abzuschließen und dadurch schnell Erfolge zu verbuchen, kann motivierend wirken und der Ungeduld des Dominanten entgegen wirken. Durch die festgesetzte WiP Grenze kann verhindert werden, dass sich der Dominante zu viel auf einmal vornimmt. Es besteht jedoch die Gefahr, dass mangels festgelegter Rollen der Dominante seine Autorität überschreitet, indem er sich über andere Teammitglieder stellt und somit die Zusammenarbeit und Harmonie des gesamten Teams gefährdet. Ach durch das Pull Prinzip besteht die Gefahr, dass sich der Dominante mit Arbeit überlädt.

\subparagraph{Scrum} Sprints können sich motivierend auswirken, da die Erreichung der festgelegten Sprint Ziele als Herausforderung und die Erfüllung als Erfolg angesehen werden kann. Die Visualisierung des Fortschritts durch ein Burndown Chart kann sich hierbei als zusätzlich motivierend auswirken. Der Schwäche, sich zu viel vorzunehmen kann in den Sprint Planing Terminen entgegengewirkt werden. Dabei ist zu beachten, dass die Abschätzung des Aufwands von Aufgaben durch das Entwicklungsteam, wie zum Beispiel mittels Planungspoker, stattfindet. Die durch Scrum fest eingeplanten Retrospektiven können die Wogen, die eventuell durch seine direkte Kommunikation entstehen, direkt geglättet werden. Ist der Hang, seine Autoritäten zu überschreiten stark ausgeprägt, könnte der Scrum Master entgegenwirken.

\subsubsection{Im Wasserfallmodell}
Durch die präzise sequenzielle Planung des Projektes sowie einer festen Projektstruktur, wird zwar der Gefahr entgegengewirkt, dass der Dominante seine Autorität überschreitet und somit Konflikte im Team entstehen. Doch wird durch diese präzise sequenzielle Planung auch die Software erst relativ spät im Entwicklungsprozess an den Kunden ausgeliefert. Dies könnte sich wegen der ungeduldigen Eigenschaft, sowie den fehlenden Teilerfolgen demotivierend auswirken. Des Weiteren würde die sequenzielle starre Planung weniger Freiraum für Risikobehaftete Ideen bringen. Dazu besteht die Gefahr, dass sich der Dominante bei der Anfangsplanung überschätzt und es dadurch zu Fehleinschätzungen im Projektplan kommt.

\subsubsection{Fazit}
Im Agilen Umfeld fühlt sich der Dominante wegen der Inkrementellen Arbeitsweise und der damit verbundenen Erfolgserlebnisse wohl. Bei einer Organisation nach Scrum könnte bei einer starken Ausprägung der autoritären Eigenschaft besser als durch Kanban entgegen gewirkt werden. Der Dominante würde sich im Wasserfallmodell, wenn er keine Leitende oder Planende Funktion inne hat eher unwohl fühlen.

\subsection{Die Initiative}
Die Initiative ist in einer Umgebung, in der sie Sichtbarkeit und Popularität im Team aber auch darüber hinaus erfährt, motiviert. Gerne hat sie Raum, um ihre Ideen einzubringen. Wenn sie dabei für ihre Ideen positives Feedback, Anerkennung oder Unterstützung Anderer erfährt, wirkt sich das besonders motivierend aus. Zu den Stärken der Initiativen gehört Kreativität und Impulsivität, was sie zu einer guten Ideengeberin macht. Außerdem kann sie mit ihrer Überzeugungskraft und Optimismus motivierend auf das ganze Team wirken, wodurch sie oft an Popularität in der Gruppe gewinnt. Diese kann sie nutzen um als Schlichterin in der Gruppe zu fungieren. Ist der Hang zur Impulsivität jedoch stark ausgeprägt, können aber auch gerne Details bei der Planung oder Durchführung von Aufgaben übersehen werden. Außerdem kann die Motivation für bereits angefangene Aufgaben verloren gehen, woran die Qualität leiden kann.

\subsubsection{Im Agilen Umfeld}
Das Projekt um motivierte Personen aufzubauen und diese zu unterstützen ist eines der Agilen Prinzipien. Dadurch wird der Initiativen der Raum gegeben, um mit ihren Ideen Andere zu überzeugen und positives Feedback zu ernten. Durch eine üblicherweise enge Zusammenarbeit zwischen Managern und Entwickler während des Projektverlaufs, hat die Initiative die Möglichkeit ihre Sichtbarkeit und Popularität über die Grenzen des Entwicklungsteams hinweg auszubauen. Des weiteren können ihre kreativen Ideen auch noch im laufe des Projektes berücksichtigt werden.

\subparagraph{Kanban} Durch den kontinuierlichen und flexiblen Ablauf bietet Kanban den nötigen Freiraum, um der Initiativen jederzeit die Möglichkeiten zu geben, ihre kreativen Ideen einzubringen. Dabei besteht Gleichzeitig die Gefahr, dass durch die impulsive und überzeugende Art übermäßig viele Umplanungen vorgenommen werden, worunter die Effizienz der Entwicklung leiden kann. Die Visualisierung des aktuellen Entwicklungsstandes auf dem gemeinsamen Kanban Board kann der Initiativen helfen, sich auf die aktuellen Arbeiten zu fokussieren. 

\subparagraph{Scrum} Durch die fest eingeplanten Scrum Events bekommt die Initiative regelmäßig den Raum, um ihre Popularität und Sichtbarkeit in der Gruppe auszubauen. Sprints könnten der Initiativen bei einer starken Ausprägung ihrer impulsiven Eigenschaft den Rahmen geben, sich auf angefangene und festgelegte Aufgaben zu konzentrieren. Jedoch könnte die starre Sprintplanung der Initiativen ein Stück weit den Raum nehmen, um jederzeit ihre kreativen Ideen einzubringen. Durch die Impulsivität und sinkende Motivation für die Fertigstellung von Aufgaben bekommt eine klare Definition of Done eine wichtige Bedeutung um einen Qualitätsmaßstab festzulegen. 

\subsubsection{Im Wasserfallmodell}
Ein klar definierter Projektplan könnte der Initiativen den Rahmen geben um sich besser auf die bevorstehenden Aufgaben zu fokussieren. Jedoch hat sie nicht den Freiraum, um ihre Ideen auch später im Projektverlauf unterzubringen. Durch die klarere Trennung zwischen Entwicklungsteam und Managern hat die Initiative außerdem eine kleinere Audienz, um mit ihren Ideen zu glänzen, was sich motivierend ausüben würde.

\subsection{Fazit}
Zusammenfassend lässt sich sagen, dass die Initiative sich in einem Agilen Umfeld wohl effizienter einsetzen lässt. Bei Kanban sollte darauf geachtet werden, dass die Initiative durch ihre Impulsivität nicht Chaos auslöst. In Scrum sollte darauf geachtet werden, dass die Initiative auch während eines geplanten Sprints den Raum hat um ihre Ideen einzubringen und Feedback dafür zu bekommen. Das Wasserfallmodell würde sich vor Allem wegen dem fest definierten Projektplan demotivierend auswirken. 


\subsection{Der Stetige}
Der Stetige wird unter Anderem durch die unterstützende Teilnahme an Aufgaben motiviert, ohne dabei aber zwingend als treibende Kraft mitzuwirken. Teammitgliedern zu helfen und dadurch Beziehungen aufzubauen bringt hierbei die Motivation. Eine Anerkennung dieser unterstützenden Art durch das Team wirkt sich hierbei besonders motivierend aus. Zu den Stärken gehört, dass sich der Stetige durch seine empathischen und freundlichen Eigenschaften, sowie der Fähigkeit sich unterzuordnen bestens dafür eignet, um die Kommunikation mit Stakeholdern zu übernehmen. Außerdem nimmt er sich gerne Zeit zuzuhören und nimmt sich auch Probleme anderer an, was ihn zu einem guten Team Player macht. Zu den Schwächen gehört eine gewisse Inflexibilität gegenüber Veränderungen. Des Weiteren besteht die Gefahr, dass sich der Stetige mit Aufgaben überlädt, in der Angst mit einem "Nein" die Harmonie des Teams zu gefährden. Zuletzt kritisiert der Stetige nicht gerne und wird auch nicht gerne kritisiert.

\subsubsection{Im Agilen}
In Agilen Projekten werden Personen und zwischenmenschliche Interaktionen über Prozesse und Tools angeordnet. Diese Priorisierung kann sich positiv auf die Harmonie im Team und somit motivierend. Das Agile Prinzip der technischen Exzellenz, wird unter Anderem durch Code Reviews sicher gestellt. Hierfür bietet sich der Stetige bestens als geduldiger Überprüfer an. Es sollte aber darauf geachtet werden, dass der Stetige berechtigte Kritik anbringt und diese nicht aus Angst die Harmonie zu gefährden zurück hält. Bei dem agilen Konzept, auf Veränderungen zu reagieren, anstatt einem Plan zu Folgen, könnte sich der Stetige jedoch bei zu spontanen Änderungen überrumpelt fühlen.

\subparagraph{Kanban} Angefangene Aufgaben möglichst schnell abzuschließen ist einer der Kernkonzepte von Kanban. Hierbei kann der Stetige durch seine unterstützende Art Anderen helfen, um Tasks abzuschließen. Die daraus entstehenden Beziehungen und Anerkennung würde sich motivierend auswirken. Außerdem wird durch das gemeinsame arbeiten an Aufgaben das Wissen in einem Team verteilt, wodurch weniger Flaschenhälse entstehen und auch zukünftige Tasks schneller abgearbeitet werden können \cite{kanban}. Durch die Fokussierung auf die aktuellen Aufgaben und weniger auf die Planung könnte dem Stetigen jedoch die notwendige Planungssicherheit fehlen.

\subparagraph{Scrum} Die Definition fester Rollen, sowie Sprints könnte dem Stetigen die notwendige Planungssicherheit und Struktur geben, um sich im Agilen Umfeld wohl zu fühlen. In den Regelmäßigen Scrum Events hätte der Stetige den Raum um Beziehungen aufzubauen.

\subsubsection{Wasserfallmodell}
Die Planungssicherheit würde dem Stetigen die Sicherheit geben, um sich wohl zu fühlen. 

\subsubsection{Fazit}
Im Gegensatz zum Dominanten und zur Initiativen würde sich der Stetige auch im Wasserfallmodell wohl fühlen. Im Agilen sollte darauf geachtet werden, dass Änderungen am Prozess und an der Arbeitsweise transparent und kontinuierlich erfolgt. Scrum könnte durch die Sprints und festen Rollen mehr Sicherheit als Kanban geben. 
 
\subsection{Die Gewissenhafte}
Die Gewissenhafte lässt sich vor Allem durch einen hohen Qualitätsstandard bei der Durchführung von Aufgaben motivieren. Zu den größten Stärken der Gewissenhaften gehört die Gründlichkeit und Ausdauer bei der Bearbeitung von Aufgaben. Auch bei den Aufgaben Anderer testet die Gewissenhafte gerne und bringt Verbesserungsvorschläge ein. Ist ihr Qualitätsbewusstsein jedoch zu stark ausgeprägt kann dies auch ein Nachteil sein. Dann "verkünstelt" sich die Gewissenhafte gerne.

\subsubsection{Im Agilen}
Das Prinzip der frühen und kontinuierlichen Auslieferung von Software kommt dem Gewissenhaften nicht so ganz entgegen. Das Prinzip der technischen Exzellenz kommt ihm entgegen. 
\subparagraph{Kanban}
In Scurm beispielsweise gibt es eine Klare Struktur wie Aufgaben mit User Stories festgehalten werden usw. ...?
\subparagraph{Scrum}
Beispielsweise bei zu lang und ausführlich gehaltenen nicht faktenbasierten Diskussionen und Meetings (arbeitet auch gerne mal alleine!). Dabei ist vor Allem bei den in Scrum üblichen Meetings zu achten. Festes Ende der Aufgaben: Entgegenwirken würde hierbei beispielsweise die festgelegte Dauer eines Sprints in Scrum

\subsection{Im Wasserfallmodell}
Aber auch die Due Dates in Wasserfallmodellen sind hartes ende vong perfektionismus her.. Des weiteren ist die Motivation der Gewissenhaften davon abhängig, in wie weit die Organisation des Projektes sowie Entscheidungen logisch, nachvollziehbar und Faktenbasiert stattfinden. Es kann sich demotivierend auswirken, wenn wichtige Entscheidungen "aus dem Bauch heraus" getroffen werden. In Wasserfallmodellen werden die Aufgaben und der Prozess oft sehr Früh und daher nicht der Realität entsprechend geplant. 

\cite{disc_pm}
\subsection{Teamzusammensetzung}
Bei einem Team also mit überwiegend X und X blablabla.

\section{Beispiele}
Wenn noch Platz ist, könnten hier Beispiele aus der Praxis oder mit fiktiven Personas beschrieben werden.

\section{Fazit}
Hier ein Fazit.

\bibliographystyle{asmems4}

% Here's where you specify the bibliography database file.
% The full file name of the bibliography database for this
% article is asme2e.bib. The name for your database is up
% to you.
\bibliography{pm}

\end{document}
